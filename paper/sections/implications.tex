\section{Implications for the Statistical Theory of Hallucinations}

The suppressor motif sharpens the consequences of Kalai et al.'s inevitability result~\cite{kalai2025why}.
First, it shows that the statistical incentive to guess is realised through concrete architectural structure.
Suppressors are not surface heuristics but deeply embedded circuits that reshape the residual stream before the rest of the network has acted.

Second, it complicates evaluation reform.
Suppressed calibration metrics improve in tandem: expected calibration error drops from $0.122$ to $0.091$, the Brier score from $0.033$ to $0.024$, and negative log-likelihood from $0.151$ to $0.113$ (Figure~\ref{fig:calibration}). Changing benchmarks to reward calibrated abstention is necessary to prevent new suppressors from forming, but already-trained models may remain stuck in hedging attractors even after the incentives shift.
Interventions must therefore operate at the circuit level---for example by steering the suppressor OV direction or regularizing its activations during fine-tuning.

Third, the motif suggests a form of learned universality.
Different architectures converge on suppressors despite differing head layouts, attention mechanisms, and tokenizers.
This supports the view of suppressors as behavioral priors: gradient descent repeatedly rediscovers them because they satisfy the conflicting optimisation objectives imposed by our datasets and evals.

\begin{figure}[t]
    \centering
    \includegraphics[width=0.45\textwidth]{figures/calibration_curve.pdf}
    \caption{Reliability diagram on the probe suite. Suppressor removal improves calibration (ECE $0.122 \rightarrow 0.091$, Brier $0.033 \rightarrow 0.024$, NLL $0.151 \rightarrow 0.113$).}
    \label{fig:calibration}
\end{figure}
