\section{Conclusion}
\label{sec:conclusion}

Mechanistic interpretability has produced remarkable insights—induction heads, IOI circuits, arithmetic subcircuits—but these discoveries relied on ad-hoc methodologies vulnerable to cherry-picking, narrow parameter sweeps, and irreproducibility. TinyLab addresses this gap by providing the first standardized, bias-resistant framework for systematic circuit discovery in transformers.

Our design enforces four key principles: (1)~standardized ablation batteries preventing ad-hoc analysis, (2)~dual-observable measurement forcing falsification across power and information metrics, (3)~extended parameter sweeps with random baselines catching narrow-range artifacts, and (4)~cross-architecture validation distinguishing universal circuits from model-specific quirks. Together, these principles transform circuit discovery from exploratory data analysis into rigorous, replicable science.

We validate TinyLab by discovering the L0 hedging coalition in GPT-2—circuits providing mechanistic evidence consistent with Kalai et al.'s prediction that binary evaluation incentivizes trading factuality for hedging. The coalition survives every methodological test: it ranks in the 99\textsuperscript{th} percentile of 1,000~random layer-0 ablations, affects both accuracy ($\Delta$LD~$+30\%$) and calibration (ECE~$-25\%$), replicates across GPT-2 Small, Medium, and Mistral-7B despite architectural differences, and exhibits quantifiable mediation (67\% through layer-11 residual stream).

This work makes three contributions. \textbf{First}, we provide complete infrastructure for reproducible circuit discovery ($\sim$1,600~lines of code, full documentation, cross-platform support, seed packs, determinism verification). \textbf{Second}, we introduce methodological innovations (H1/H5/H6/H7~batteries, dual observables, random baseline enforcement, cross-architecture pipelines) that catch the narrow-sweep problem invalidating our prior work. \textbf{Third}, we demonstrate validation via the hedging coalition, proving TinyLab surfaces genuine behavioral circuits learned by gradient descent, not artifacts of ad-hoc analysis.

By treating mechanistic interpretability as an engineering discipline requiring standardized tooling—not just conceptual insight—TinyLab enables the systematic, reproducible science needed to understand and align increasingly powerful language models. The framework, datasets, and complete reproducibility package are available at \url{https://github.com/username/tinyLab}, designed for drop-in replication across model families and community extension.

Future work will expand model coverage (Llama, Pythia, Qwen), instrument training dynamics (when do coalitions emerge?), integrate SAE-based feature decomposition (H7~full implementation), and build an open database of circuits with standardized measurements. TinyLab provides the foundation for this circuit taxonomy, transforming one-off discoveries into cumulative, falsifiable knowledge about how transformers compute.
