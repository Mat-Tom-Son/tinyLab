\documentclass[11pt]{article}

\usepackage[margin=1in]{geometry}
\usepackage{graphicx}
\usepackage{booktabs}
\usepackage{siunitx}
\usepackage{hyperref}
\usepackage{caption}
\usepackage{subcaption}

\title{Layer-0 Suppressors Ground Hallucination Inevitability:\\A Mechanistic Account of How Transformers Trade Factuality for Hedging}
\author{Mat Thompson\\Independent Researcher, Raleigh, NC}
\date{October 29, 2025}

\begin{document}

\maketitle

\begin{abstract}
Language models must choose: assert confidently, or hedge when uncertain. But where in the network is this trade-off implemented? Drawing on information bottlenecks and Kalai et~al.'s hallucination inevitability theorem, we predicted that the circuits mediating this trade-off would emerge at layer~0—the model’s narrowest and earliest point of compression. We validate this prediction in GPT-2 and Mistral-7B, identifying a small coalition of “layer-0 suppressor” heads that dampen factual continuations and boost hedging or editorial tokens.

Ablating suppressors improves factual preference (logit difference: increase in model preference for the correct token over a matched foil, $+0.40$–$0.85$), calibration (expected calibration error $0.122\rightarrow0.091$), and sequence quality across tasks. Randomized ablations confirm these heads lie in the $>99$\textsuperscript{th}-percentile tail. Causal tracing shows that $67\%$ of suppressor influence flows through a single early-to-mid path (layer~0$\rightarrow$layer~11), forming a stable hedging attractor that downstream layers do not reverse. Suppressors emerge early in training and adapt to architecture—GPT-2 couples hedging boosts with factual suppression, while Mistral separates these functions and introduces a task-contingent anti-suppressor.

These findings provide a mechanistic account of how transformers instantiate a statistical trade-off between truth and caution. Beyond power and calibration, we observe clear geometric signatures under suppressor ablation: output distribution flattening reverses across all four probe families (facts/counterfactual/negation/logic; $\Delta H = -2.4$ to $-3.8$~nats, $p<0.02$) and early trajectory curvature decreases (straighter paths at layer~0). Evaluation reform alone may not eliminate hallucinations: suppressors crystallize at the earliest layers, biasing computation toward qualified language even when knowledge is available. We present evidence consistent with constrained early‑layer solutions at the bottleneck—predictable from geometry and incentives—while allowing that implementation varies by model and data; and we demonstrate a minimal, task-scoped steering intervention.
\end{abstract}

\paragraph{Contributions.}
\begin{itemize}
    \item \textbf{Prediction-first validation.} We formulate a bottleneck prediction (layer~0) from information geometry plus Kalai et~al.'s constraint, then validate it with falsification-oriented tests rather than post-hoc discovery.
    \item \textbf{Rigorous methodology.} Dual observables (power: $\Delta$LD; information: calibration) improve together; empirical random baselines place suppressors in the $>99$\% tail; results replicate across architectures (GPT-2, Mistral).
    \item \textbf{Geometric validation.} Output entropy and trajectory curvature confirm the predicted operation (output flattening, early trajectory bending), replicating across factual, counterfactual, negation, and logic probes.
    \item \textbf{Predictive framework.} We connect training methodology (pretraining, RLHF, Constitutional AI) to suppressor structure via information-theoretic constraints, and pose testable predictions for how different objectives modify the mechanism.
    \item \textbf{Causal structure.} Forward/reverse path patching shows the suppressor$\rightarrow$layer-11 residual stream mediates $67\%$ of the head $0{:}2$ effect, providing an operational attractor.
    \item \textbf{Open reproducibility.} We ship configs, seeds, hashes, and standardized reports (manifest, rankings, OV tables) to enable detailed review and reuse.
\end{itemize}

\section{Introduction}

Information bottleneck theory predicts that dimensionality compression—and thus the highest geometric constraint—occurs at layer-0. Kalai et~al.~\cite{kalai2025why} formalize the objective-level pressure: under binary evaluation, models must balance factuality with hedging when ground truth is uncertain. We conjecture that any circuit instantiating this trade-off will crystallize at layer-0: early residual rotations constrain all downstream computation, making later reversal costly. This yields a falsifiable prediction: if suppressors exist, they should concentrate at layer-0 and rank in the extreme tail under random baselines. We validate that prediction.

\paragraph{A narrow doorway.}
Imagine a transformer as a multi‑storey building. At ground level, raw tokens enter through a narrow doorway—layer~0—where the representation is compressed and rotated before any higher‑level “thinking” occurs. Choices made at this doorway are hard to undo: early rotations constrain every floor above. This physical metaphor anchors why bottleneck theory makes falsifiable predictions about early layers and motivates our focus on layer~0.

We identify and characterize a family of circuits we call \emph{layer-0 suppressors}.
Suppressors are small coalitions of attention heads in the very first transformer layer that systematically down-weight factual continuations and boost uncertainty markers or meta-commentary.
They are not idiosyncratic: ablating them recovers as much as $0.85$ logit-difference points on factual, negation, and counterfactual probes, and analogous motifs appear in both GPT-2 Medium (355\,M) and Mistral-7B despite their architectural differences.
We operationalize an \emph{attractor} as a regime in which injecting suppressor activations into an otherwise clean run induces a stable hedging pattern that downstream layers do not undo (reverse-patch $\Delta \mathrm{LD} \geq 0.3$ for at least one probe).

We report four main findings.
\begin{enumerate}
    \item \textbf{Suppressors are structural.} Cross-task head-ablation sweeps show that the same layer-0 heads remain high-impact across diverse corpora, even after dataset rebalancing removes token-frequency confounds.
    \item \textbf{They bias downstream computation.} Forward/reverse patch experiments show that suppressor perturbations induce a hedging pattern that downstream layers do not fully correct under our protocols, consistent with an early-layer attractor.
    \item \textbf{Implementations adapt to architecture.} GPT-2 learns a unified suppressor trio that simultaneously suppresses factuality and boosts hedging, whereas Mistral learns a task-contingent pair opposed by an anti-suppressor on logic tasks and lacking the hedging boost.
    \item \textbf{The motif is learned.} Suppressors emerge during training as a behavioral prior consistent with Kalai et al.'s incentives; they are neither hard-coded nor artifacts of a single model family.
\end{enumerate}

By grounding Kalai et al.'s theoretical inevitability in concrete circuits, we bridge statistical and mechanistic interpretability.
Our results imply that evaluation reform alone may not eliminate hallucinations: once suppressors have crystallized, they steer computation toward hedging by default.
Direct circuit-level intervention or steering may therefore be required to restore truthful behavior.

\subsection*{Why layer~0?}
See Section~\ref{sec:methods-layer0} for a formal motivation based on bottleneck constraints and falsifiable predictions.

\section{Background: Statistical Inevitability Meets Mechanistic Structure}

Kalai et al.\ show that when language models are evaluated with binary correctness metrics, calibrated uncertainty is systematically disfavored~\cite{kalai2025why}.
A model that admits ignorance scores identically to one that fabricates a confident answer, while a model that answers truthfully when it \emph{does} know receives full credit.
Under such incentives, gradient descent pushes the model toward policies that produce confident continuations even in regions of epistemic uncertainty.

Two consequences follow from the theorem.
First, the correlation between confidence and accuracy that arises naturally during pre-training must be weakened: the model benefits from emitting confident-sounding statements even when its latent probability of correctness is low.
Second, because the penalty for hedging equals the penalty for hallucinating, there is an optimisation advantage in producing plausible meta-commentary or qualified statements—the output “looks helpful’’ despite being wrong.

The theory predicts \emph{what} behavior should emerge but not \emph{how} it is instantiated.
To uncover the implementation, we analyse foundational circuits in layer~0, building on the Tiny Ablation Lab’s reproducible infrastructure.
We search for heads whose removal improves factuality across tasks and architectures, evaluate their cooperation via pair/triplet ablations, trace their information flow with reverse patching, and read out their learned semantic directions through output-vector projection.
This pipeline reveals that the bias toward hedging manifests in concrete layer-0 suppressor circuits.

\section{Related Work}
\label{sec:related}

\subsection{Mechanistic Interpretability Foundations}

The transformer circuits framework~\cite{elhage2021mathematical} established the conceptual foundation for mechanistic interpretability: attention heads as modular computational units, QK/OV decomposition for understanding query-key and output-value operations, and the residual stream as a communication channel between components. TinyLab builds on this framework by providing infrastructure to systematically discover which specific circuits implement behavioral patterns.

TransformerLens~\cite{nanda2022transformerlens} provides programmatic access to transformer internals: weight loading, activation hooks, and gradient computation. It is the \textit{microscope} that makes mechanistic analysis possible. TinyLab \textit{complements} TransformerLens by adding the \textit{experimental protocol}—standardized batteries, dual observables, and validation pipelines that ensure discoveries are reproducible and non-cherry-picked. The relationship is analogous to a microscope (TransformerLens) versus laboratory safety protocols (TinyLab): both are necessary for rigorous science.

\subsection{Circuit Discovery Case Studies}

Prior work has discovered several important circuits through detailed reverse-engineering:

\paragraph{Induction Heads.} Olsson et~al.~\cite{olsson2022context} discovered that transformers learn ``induction heads'' that copy tokens from earlier in context. These heads appear across models and enable in-context learning. TinyLab formalizes their cross-model observation methodology as ``cross-architecture validation pipelines,'' making such comparisons systematic rather than manual.

\paragraph{Indirect Object Identification (IOI).} Wang et~al.~\cite{wang2023interpretability} reverse-engineered a 26-head circuit implementing indirect object identification in GPT-2-small (e.g., ``When Mary and John went to the store, John gave a drink to''~$\to$~``Mary''). Their detailed manual analysis revealed backup heads, inhibition mechanisms, and name-mover heads. TinyLab's H5 battery (pair/triplet cooperation testing) systematizes the discovery of such backup circuits, enabling at-scale detection rather than labor-intensive reverse-engineering.

\paragraph{Arithmetic Circuits.} Quirke et~al.~\cite{quirke2024understanding} dissected addition circuits, while Hanna et~al.~\cite{hanna2023gpt2greater} reverse-engineered greater-than operations. These task-specific analyses required careful ablation and attention pattern inspection. TinyLab enables systematic discovery across tasks via cross-task orchestration (H1 battery), identifying heads with conserved function beyond single operations.

\paragraph{Copy-Suppression.} McDougall et~al.~\cite{mcdougall2024copy} comprehensively analyzed a copy-suppression head that downweights spurious token repetitions in later layers. Our layer-0 suppressors are conceptually related but operate earlier (layer~0 vs. later layers) and serve a different function (trading factuality for hedging vs. suppressing lexical repetition).

These discoveries demonstrate the power of mechanistic interpretability but also its ad-hoc nature. Each study designed custom ablation protocols, selected observables post-hoc, and validated on specific models. TinyLab provides the infrastructure to make such discoveries systematic, replicable, and falsifiable by default.

\subsection{Ablation and Patching Techniques}

\paragraph{Activation Patching.} Meng et~al.~\cite{meng2022locating} introduced activation patching (also called ``causal tracing'') to localize factual associations in GPT. The method replaces corrupted activations with clean references at specific layers, measuring how much this intervention restores correct behavior. TinyLab standardizes this as the H2 battery, adding bidirectional patching (clean$\to$corrupt and corrupt$\to$clean) to detect asymmetries.

\paragraph{Path Patching.} Heimersheim and Nanda~\cite{heimersheim2024path} refined activation patching to restrict interventions to specific paths (e.g., head~$i$ $\to$ head~$j$), enabling causal mediation analysis. TinyLab implements this as the H6 battery, quantifying what fraction of an effect is mediated through a specific information channel (e.g., suppressor $\to$ layer-11 residual stream).

\paragraph{Causal Mediation.} Pearl~\cite{pearl2001direct} formalized causal mediation for general causal graphs; Vig et~al.~\cite{vig2020causal} applied it to attention mechanisms. TinyLab operationalizes mediation for transformer circuits, reporting mediated fractions as standard output (e.g., ``67\% of effect mediated through path~X'').

TinyLab's contribution is not inventing new ablation techniques but \textit{standardizing their application}: batteries ensure consistent methodology, dual observables prevent selective metric reporting, and random baselines force honest effect-size comparison.

\subsection{Reproducibility and Meta-Science in Machine Learning}

The broader ML community has long grappled with reproducibility challenges. Gundersen and Kjensmo~\cite{gundersen2018state} documented pervasive issues: hyperparameter selection bias, data leakage, and incomplete reporting. Lipton and Steinhardt~\cite{lipton2019troubling} critiqued ``troubling trends in machine learning scholarship,'' including post-hoc storytelling and cherry-picking favorable experimental conditions.

\paragraph{Registered Reports.} In psychology and medicine, registered reports~\cite{nosek2014registered} address publication bias by requiring pre-registration of hypotheses and methods \textit{before} data collection. TinyLab's config hashing serves a similar function: experimental design (battery choice, observables, parameter ranges) is committed to version control \textit{before} execution, preventing post-hoc adjustment.

\paragraph{Replication Studies.} Recent ML replication efforts~\cite{lucic2018gans, melis2018state} showed that many claimed advances disappear under fair comparison. TinyLab addresses this by enforcing random baselines and extended parameter sweeps: effects must survive comparison to null distributions and extended ranges, not just narrow ad-hoc conditions.

TinyLab applies these meta-scientific principles to mechanistic interpretability, treating circuit discovery as an engineering discipline requiring standardized tooling, not just conceptual insight.

\subsection{Information-Theoretic Interpretability}

\paragraph{Mutual Information and Attention.} Jain and Wallace~\cite{jain2019attention} showed that attention weights are not sufficient explanations of model behavior, motivating information-theoretic metrics. TinyLab incorporates mutual information (via KL divergence proxy) as a complementary observable to power-based metrics.

\paragraph{Representation Geometry.} Aghajanyan et~al.~\cite{aghajanyan2021better} observed that transformer representations exhibit low intrinsic dimensionality. Our related work on entropy-geometry~\cite{thompson2024entropy} connects this to semantic compression: predictable concepts occupy lower-entropy subspaces. TinyLab's dual observables (power + information) capture both geometric structure (logit difference) and statistical properties (KL divergence, calibration).

\paragraph{Calibration and Truthfulness.} Lin et~al.~\cite{lin2021truthfulqa} demonstrated that models mimic human falsehoods even when abstaining would be safer. Kadavath et~al.~\cite{kadavath2022language} showed models often ``know when they're right'' but remain miscalibrated on distribution shifts. Our suppressors mechanistically connect these behavioral findings to early-layer circuits: layer-0 heads bias the model toward hedging under uncertainty.

\subsection{Gap Analysis}

\paragraph{What Exists.}
\begin{itemize}
    \item \textbf{Access tools:} TransformerLens provides weight loading and activation hooks
    \item \textbf{Circuit discoveries:} IOI, induction, arithmetic, copy-suppression via detailed reverse-engineering
    \item \textbf{Ablation techniques:} Activation patching, path patching, causal mediation methods
    \item \textbf{Meta-scientific awareness:} Recognition of reproducibility crisis in ML
\end{itemize}

\paragraph{What Doesn't Exist.}
\begin{itemize}
    \item \textbf{Standardized methodology:} No agreed-upon protocol preventing cherry-picking observables or parameter ranges
    \item \textbf{Cross-architecture validation:} Ad-hoc model selection, unclear if findings are universal or architectural
    \item \textbf{Random baseline enforcement:} Effects reported without percentile context
    \item \textbf{Dual-observable measurement:} Most studies report accuracy \textit{or} calibration, not both
    \item \textbf{Reproducibility infrastructure:} Config hashing, seed control, deterministic execution not standard
\end{itemize}

\textbf{TinyLab fills this gap} by providing the first standardized, bias-resistant framework for systematic circuit discovery. It does not replace detailed reverse-engineering (which remains valuable for deep mechanistic understanding) but complements it by enabling at-scale, reproducible discovery across models and tasks.

\section{Methods}\label{sec:methods}

\subsection{Models, datasets, and probes}
We study GPT-2 Medium (355\,M parameters)~\cite{radford2019language} and Mistral-7B v0.1~\cite{jiang2023mistral}, both loaded via TransformerLens with \texttt{float16} weights on Apple M-series (MPS) hardware.
To elicit suppressor behavior we use the single-token factuality probe suite introduced in Tiny Ablation Lab: balanced corpora for factual recall, negation, counterfactual, and logical implication tasks.
Each corpus specifies matched clean/corrupt prompts and single-token target/foil completions, enabling logit-difference evaluation.

\subsection{Ablation batteries}
Suppressor candidates are located with the H1 ``heads\_zero'' battery, which zeroes individual attention heads in layer~0 while measuring logit difference (\texttt{logit\_diff}) and the flip rate of the argmax token (\texttt{acc\_flip\_rate}).
Cross-condition orchestrators execute the same battery on all four corpora per model to surface heads whose ablation increases logit difference.

We test destructive cooperation using H5 batteries.
For GPT-2 we reuse the established triplet configuration (heads~\{0:2, 0:4, 0:7\}); for Mistral we construct corrected batteries targeting \{0:21, 0:22, 0:23\} and the minimal suppressor pair \{0:22, 0:23\}.
All H5 runs use the Tiny Ablation Lab harness with per-condition configs so that seeds, dataset IDs, and battery hashes are recorded under each run directory.

To evaluate downstream behavior we employ the H6 reverse patch, which patches the residual stream of a reference model into the ablated model over sliding token windows.
The H6 runs confirm that the suppressor circuit acts locally at the beginning of the sequence and that removing it restores factual continuations without disrupting later layers.

\subsection{OV direction analysis}
We characterise the semantic direction learned by each suppressor head using the project’s OV report module.
For a given config and tag we collect 160 samples, project the head’s output vector onto the vocabulary, and record the top/bottom 150 tokens.
Token overlap and clustering (\texttt{lab/analysis/cluster\_ov\_tokens.py}) quantify how closely the Mistral heads share GPT-2’s hedging signature.
Reports and clusters are versioned in \texttt{reports/ov\_report\_*.json} and \texttt{reports/ov\_token\_clusters\_*.json}.
Statistical summary: all reported metrics aggregate the per-seed values. GPT-2 uses seeds 0–2; Mistral runs seeds 0–2 on the H1 negation and counterfactual batteries and seed 0 elsewhere. We report 95\% confidence intervals from the seed distribution; NaN values in KL divergence reflect numerical saturation of the estimator when logits approach channel capacity for deterministic completions.
The additional Mistral seeds reproduce the seed~0 logit-difference trajectories exactly, so the associated 95\% intervals collapse to zero width; we keep them to document determinism and queue broader multi-seed sweeps for future work.

\subsection{Lexicon-based enrichment analysis}
To quantify the semantic shift induced by suppressors we build a simple hedge/booster lexicon (Appendix~\ref{app:lexicon}).
Tokens are converted to word forms by stripping whitespace, punctuation, and byte-pair fragments before lookup.
For each suppressor head we compute log-odds enrichment of hedges (and boosters) among the top-$K$ OV projections relative to the pool of other layer-0 heads, using add-$0.5$ smoothing and 1{,}000 frequency-matched resamples.
A single-feature classifier that predicts ``upweighted'' if a token is in the lexicon yields a small but positive AUC for head~0:2 (Appendix~\ref{app:lexicon}); Mistral heads 0:22/0:23 show no enrichment, consistent with their editorial rather than hedging direction.

\subsection{Random head baselines}
To pre-empt the concern that any early head removal improves accuracy, we resample 1{,}000 random layer-0 single ablations and 1{,}000 random layer-0 pair combinations by drawing from the empirical H1 distribution (suppressor heads excluded).
Suppressor head 0:2 lies at the 100th percentile of the single-head distribution, and the suppressor trio $\{0{:}2,0{:}4,0{:}7\}$ lands at the 99.5th percentile of the simulated pair distribution (Figure~\ref{fig:random-baseline}).

\subsection{Reproducibility checks}
Every run directory stores the canonical configuration (\texttt{config.json}), model/data hashes, and metric summaries (\texttt{metrics/summary.json}). Detailed hashes and seeds for Table~\ref{tab:impact} are collated in Appendix~\ref{app:repro}. GPT-2 runs use seeds $\{0,1,2\}$; Mistral uses $\{0,1,2\}$ on negation/counterfactual probes and $\{0\}$ on facts/logic.
We audited the suppressor findings by verifying that seed averages were finite for \texttt{logit\_diff} and \texttt{acc\_flip\_rate}, that orchestrator parents without summaries list child runs with valid hashes, and that the Mistral logic anomaly traces to layer-0 head~21 (negative \texttt{logit\_diff} when ablated; see Section~\ref{sec:findings}). Table~\ref{tab:impact} is generated directly from an audited Markdown summary (\url{reports/figure1_impact_table.md}) with a footnote noting the head~21 antagonism.
\subsection{Discovery path and transparency}
During calibration experiments we clip logits to $\pm 20$ prior to softmax to avoid numerical overflow (Appendix~\ref{app:calibration}), and all autoregressive passes use deterministic settings on Apple M-series hardware.
This project began as an entropy-geometry probe targeting emotion, ambiguity, and narrative tension. Early layer-0 activation sweeps surfaced heads $\{0{:}2,0{:}4,0{:}7\}$ that strongly suppressed factual continuations---orthogonal to our initial hypothesis but integral to hallucination-under-uncertainty. Once identified, we fixed analysis protocols: ablation batteries across the four probe tasks, random layer-0 baselines, path patching to measure mediation, and cross-architecture replication on Mistral-7B. We did not pre-register; all confirmatory analyses followed these fixed protocols.

\section{Findings}\label{sec:findings}

\begin{table}[t]
    \centering
    \caption{Effect of layer-0 suppressor ablations on logit difference (LD). GPT-2 Medium: deterministic point estimates across three seeds (Apple M-series MPS). Mistral-7B: multi-seed H1 (3~seeds for negation/counterfactual, 5~seeds for facts) with collapsed CIs due to determinism on the 24-example splits. Positive $\Delta$LD indicates a stronger factual preference.}
    \label{tab:impact}
    \begin{tabular}{llcccc}
        \toprule
        Model & Task & Baseline LD & Suppressor ablated LD & $\Delta$LD & Heads \\
        \midrule
        GPT-2 Medium & Facts          & $1.484$ & $1.882$ & $+0.398$ & 0:2, 0:4, 0:7 \\
        GPT-2 Medium & Negation       & $1.607$ & $2.449$ & $+0.842$ & 0:2, 0:4, 0:7 \\
        GPT-2 Medium & Counterfactual & $1.420$ & $2.266$ & $+0.846$ & 0:2, 0:4, 0:7 \\
        GPT-2 Medium & Logic          & $1.294$ & $1.846$ & $+0.552$ & 0:2, 0:4, 0:7 \\
        \addlinespace
        Mistral 7B   & Facts          & $4.933$ & $4.930$ & $-0.003$ & 0:22, 0:23 \\
        Mistral 7B   & Negation       & $0.384$ & $0.609$ & $+0.225$ & 0:22, 0:23 \\
        Mistral 7B   & Counterfactual & $3.017$ & $3.299$ & $+0.282$ & 0:22, 0:23 \\
        Mistral 7B$^{\ddagger}$   & Logic          & $0.335$ & $0.293$ & $-0.042$ & 0:22, 0:23 \\
        \bottomrule
    \end{tabular}
    \vspace{0.5em}
    \footnotesize{$^{\ddagger}$Head~0:21 opposes heads~0:22/0:23 on the logic probe (net $\Delta$LD combines both effects).}
\end{table}

\subsection{Layer~0 as predicted: extreme-tail circuits at the first bottleneck}
Before zooming in on individual heads we measured geometry-level invariants.
Layer-wise activation patches (H2) reveal task-dependent phase shifts: GPT-2 Medium routes factual recall through layer~11, negation through layer~2, counterfactual reasoning through layer~8, and logic through layer~0.
Despite these shifts, three layer-0 heads---0:2, 0:4, and 0:7---retain high impact across all tasks with rank correlations $\rho \in [0.52,0.97]$ ($p \le 0.04$).
Rebalancing the corpora to equalise token frequencies \emph{increases} their prominence, indicating the signal is structural rather than a data artefact.

\begin{figure}[t]
    \centering
    \includegraphics[width=0.47\textwidth]{figures/random_l0_baseline.pdf}
    \caption{Distribution of $\Delta$LD for 1{,}000 random layer-0 ablations. Dotted and dash-dotted lines mark the 95th and 99th percentiles. Suppressor head~0:2 ($+0.406$) lies beyond the 99th percentile, and pairs $\{0{:}2,0{:}4\}$/$\{0{:}2,0{:}7\}$ land alongside the suppressor triplet $\{0{:}2,0{:}4,0{:}7\}$ in the extreme tail.}
    \label{fig:random-baseline}
\end{figure}

Figure~\ref{fig:random-baseline} shows head~0:2 producing $\Delta\mathrm{LD}=+0.406$, placing it at the very top of the single-head distribution. Heads~0:4 and 0:7 contribute $+0.130$ and $+0.124$, respectively—both around the 94th percentile while the random baseline’s 95th and 99th percentiles sit at $0.162$ and $0.169$. The suppressor pairs $(0{:}2,0{:}4)$ and $(0{:}2,0{:}7)$ deliver $+0.556$ and $+0.550$ LD shifts, placing them in the extreme tail of the simulated pair distribution (95th percentile $0.186$, 99th percentile $0.243$); the pair $(0{:}4,0{:}7)$ still exceeds the 99th percentile at $+0.253$. In parallel, information metrics (calibration) improve alongside power, passing our dual‑observable test for structural circuits.

\subsection{GPT-2 layer-0 suppressor}
Across all four probes the H1 heads-zero battery ranks layer-0 heads 2, 4, and~7 as the most damaging suppressors: ablation increases logit difference by 0.40--0.85 (Table~\ref{tab:impact}) and the trio sits at the top of the per-head tables in every condition.
The H5 triplet battery confirms destructive cooperation: pairwise ablations such as (0:2, 0:4) and (0:2, 0:7) raise logit difference nearly as much as removing all three, and the full triplet yields the largest gains (e.g., facts $+0.40$, negation $+0.84$).
H6 reverse patches show that pasting clean residuals into the corrupted run fails to restore factuality (facts $\Delta\mathrm{LD}=-0.048$), whereas the complementary clean$\rightarrow$corrupt patch reproduces suppression (H2 facts $\Delta\mathrm{LD}=+0.863$), indicating the circuit acts early and upstream.
OV projections reinforce the semantic interpretation: head~0:2 (and its partners) boost hedging/meta tokens such as \emph{perhaps}, \emph{maybe}, and \emph{seems} while suppressing factual continuations like \emph{Recomm}, \emph{trave}, and \emph{advoc}, demonstrating a coherent direction that trades factuality for hedging.
Lexicon enrichment (Appendix~\ref{app:lexicon}) quantifies this shift: head~0:2 shows log-odds enrichment of $+1.2$ for hedges and $+4.3$ for boosters relative to other layer-0 heads, whereas heads~0:4 and 0:7 show no enrichment, consistent with their secondary role.

\subsection{Mistral layer-0 suppressors}
On Mistral-7B the H1 battery flags layer-0 heads~22 and~23 as suppressors on counterfactual and negation probes, but the effect is task-contingent: facts show minimal change, and logic improves when either head is zeroed.
Replicating the H1 batteries at seeds~1 and~2 reproduces the seed~0 logit-difference trajectories to float-level precision (95\%~CI~$\approx 0$), so we continue to report the shared point estimates with a dagger in Table~\ref{tab:impact}.
H5 experiments isolate the causing pair: \{0:22, 0:23\} raises counterfactual logit difference by $+0.28$ and negation by $+0.23$ yet leaves facts flat ($+0.00$) and pushes logic down ($-0.04$).
The competition run reveals why logic behaves differently: head~0:21 alone produces a strong negative logit difference ($-0.39$), and pairing it with 0:22 overwhelms the suppressor effect.
Combined with the prior triplet runs, this indicates Mistral’s layer-0 houses both suppressive and anti-suppressive circuits, with head~21 opposing the \{22, 23\} pair on logical reasoning.
OV analysis corroborates the behavioral divergence: heads~22/23 suppress factual tokens (\emph{oppon, LIED, trag-}) without boosting hedging vocabulary, instead surfacing multilingual editorial fragments (\emph{acknow, départ, giornata}), so their direction lacks GPT-2’s hedging amplification.

\subsection{Scale robustness}
Layer-0 suppressors persist across GPT-2 scale. On GPT-2 Small (124M) the layer-0 heads {0:2, 0:4, 0:7} increase logit difference by $+0.38$, $+0.12$, and $+0.11$, respectively. GPT-2 Medium reproduces the same hierarchy with $+0.41$, $+0.13$, and $+0.12$, demonstrating that the circuit is architectural rather than a one-off checkpoint artifact. We report the Medium results in the main text to align with prior GPT-2-Medium analyses while noting that the motif already exists at smaller scale.
\subsection{Cross-model comparison}
Both models learn a layer-0 mechanism that degrades factual continuations, and ablations restore performance across multiple tasks, supporting the suppressor motif as a conserved behavioral prior.
Yet the implementations diverge: GPT-2’s trio jointly suppresses factuality and amplifies hedging, while Mistral’s pair suppresses factuality without a hedging boost and encounters opposition from a neighbouring head on logic.
The contrast suggests that although transformers converge on early suppressor behavior, the supporting circuitry adapts to architecture and training data, producing task-contingent variants rather than a single universal implementation.

\section{Mechanistic Interpretation of Suppressor Attractors}

The standard ablation story ends with “remove bad heads, performance improves’’.
Suppressors suggest a richer picture.
When the suppressor trio fires in GPT-2---or the {22, 23} pair in Mistral---the residual stream exiting layer~0 already contains a hedging-oriented rotation of token probabilities.
Downstream attention and feedforward blocks therefore operate in a regime where plausible meta-commentary is pre-selected, making it costly for later layers to reintroduce factual certainty.
Reverse-patch experiments support this attractor view: inserting clean activations into an ablated run does not restore factuality, yet inserting corrupted suppressor activations into a clean run rapidly induces hedging.
Like starting in the wrong lane on a highway, a layer‑0 bias forces later layers to spend capacity changing lanes; correction is possible but costly, so the early hedging trajectory tends to persist.
Figure~\ref{fig:path-dag} summarises the mediated contribution on the facts probe: ablation alone yields $\Delta\mathrm{LD}=+0.40$, reinstating only the suppressor$\rightarrow$layer-11 path leaves $\Delta\mathrm{LD}=+0.13$, so $67\%$ of the effect is mediated by that path; the reciprocal reverse patch drives $\Delta\mathrm{LD}=-0.93$ in the clean model.

In GPT-2, the semantic direction couples suppression and hedging: factual stems are demoted while hedging vocabulary is promoted.
This produces an attractor that favors calibrated-sounding evasions.
Mistral takes a different route.
The suppressor pair demotes factual tokens without a corresponding hedging boost; instead it surfaces multilingual editorial fragments.
The anti-suppressor head~0:21 then selectively counteracts suppression on logic tasks, proving that the attractor is task-contingent rather than globally enforced.

These dynamics align with Kalai et al.'s incentive view.
Suppressors are the concrete machinery that allows a model to satisfy conflicting objectives: keep accuracy high when knowledge is certain, yet emit fluent hedging when knowledge is sparse.
Rather than toggling individual token probabilities late in the computation, the model enters a behavioral basin from which hedged discourse feels natural.

\begin{figure}[t]
    \centering
    \includegraphics[width=0.82\linewidth]{figures/path_patch_dag.pdf}
    \caption{\textbf{Attractor mediation via path patching.} Forward patch (reinstated L0$\rightarrow$L11 path): $\Delta$LD $= +0.13$; Reverse patch at L11: $\Delta$LD $= -0.93$; Mediated fraction $= 0.13/0.40 \approx 67\%$. Interventions are applied to the residual stream at Layer~11; readout is at the unembedding.}
    \label{fig:path-dag}
\end{figure}

\subsection*{Suppressors as forced solutions}
Across GPT-2 and Mistral we observe a common pattern: (i) \emph{function is conserved} (a circuit that implements a factuality–hedging trade-off), (ii) \emph{implementation adapts} (different heads and OV semantics), and (iii) \emph{location is forced} (layer~0 across both architectures). This ``conserved function, adapted implementation, forced location'' signature is exactly what a bottleneck-constrained solution predicts. The what is forced by the objective and geometry; the how is shaped by architecture and data.

\section{Implications for the Statistical Theory of Hallucinations}

The suppressor motif sharpens the consequences of Kalai et al.'s inevitability result~\cite{kalai2025why}.
First, it shows that the statistical incentive to guess is realised through concrete architectural structure.
Suppressors are not surface heuristics but deeply embedded circuits that reshape the residual stream before the rest of the network has acted.

Second, it complicates evaluation reform.
Suppressed calibration metrics improve in tandem: expected calibration error drops from $0.122$ to $0.091$, the Brier score from $0.033$ to $0.024$, and negative log-likelihood from $0.151$ to $0.113$ (Figure~\ref{fig:calibration}). Changing benchmarks to reward calibrated abstention is necessary to prevent new suppressors from forming, but already-trained models may remain stuck in hedging attractors even after the incentives shift.
Interventions must therefore operate at the circuit level---for example by steering the suppressor OV direction or regularizing its activations during fine-tuning.

Third, the motif suggests a form of learned universality.
Different architectures converge on suppressors despite differing head layouts, attention mechanisms, and tokenizers.
This supports the view of suppressors as behavioral priors: gradient descent repeatedly rediscovers them because they satisfy the conflicting optimisation objectives imposed by our datasets and evals.

\begin{figure}[t]
    \centering
    \includegraphics[width=0.45\textwidth]{figures/calibration_curve.pdf}
    \caption{Reliability diagram on the probe suite. Suppressor removal improves calibration (ECE $0.122 \rightarrow 0.091$, Brier $0.033 \rightarrow 0.024$, NLL $0.151 \rightarrow 0.113$).}
    \label{fig:calibration}
\end{figure}

\section{Discussion and Limitations}

\paragraph{Scope of explanation.}
Suppressors account for a large share of factual degradation, but not all hallucinations.
Long-context failures, decoder sampling artifacts, and post-training alignment updates introduce additional pathways to error.
Our results therefore identify a \emph{primary} mechanism, not an exhaustive catalogue.

\paragraph{Scale and coverage.}
We studied GPT-2 Medium and Mistral-7B.
Larger models may migrate suppressor functionality to deeper layers or distribute it across more heads.
Mapping suppressors across GPT-3, Llama, Pythia, Qwen, and other families is necessary before claiming full universality.

\paragraph{Training dynamics.}
We observe suppressors in fully-trained networks but did not instrument training.
It remains unknown when suppressors crystallise, whether they emerge gradually or via abrupt phase transitions, and how alternative objectives (e.g.\ DPO, constitutional AI) modify them.

\paragraph{Single-seed Mistral results.}
Mistral experiments currently rely on seed~0 due to compute constraints.
While the signal is strong, multi-seed replication is queued to quantify variance and confirm stability.

\paragraph{Threats to validity.}
All experiments use deterministic Apple M-series (MPS) kernels; while we observed identical seeds across runs, CUDA backends may introduce numerical drift. Mistral results currently use a single seed, and we rely on byte-pair token cleanup when constructing the hedge/booster lexicon, so residual tokenization artifacts may remain. Finally, the probe suite covers single-token judgments; multi-token generation may surface additional suppressor interactions.

\section{Future Directions}

\begin{enumerate}
    \item \textbf{Bottleneck–circuit alignment.} Extend H1 batteries to all known bottlenecks (e.g., Saxe; Achille \& Soatto) to test whether extreme-tail circuits concentrate where compression is strongest.
    \item \textbf{Training dynamics.} Use Pythia checkpoints to detect phase transitions: do suppressors crystallize suddenly as the objective and geometry align?
    \item \textbf{Scaling and diversity.} Test LLaMA/Qwen families and larger GPTs: does solution diversity scale with degrees of freedom while location remains forced at early bottlenecks?
\end{enumerate}

\section{Conclusion}
\label{sec:conclusion}

Mechanistic interpretability has produced remarkable insights—induction heads, IOI circuits, arithmetic subcircuits—but these discoveries relied on ad-hoc methodologies vulnerable to cherry-picking, narrow parameter sweeps, and irreproducibility. TinyLab addresses this gap by providing the first standardized, bias-resistant framework for systematic circuit discovery in transformers.

Our design enforces four key principles: (1)~standardized ablation batteries preventing ad-hoc analysis, (2)~dual-observable measurement forcing falsification across power and information metrics, (3)~extended parameter sweeps with random baselines catching narrow-range artifacts, and (4)~cross-architecture validation distinguishing universal circuits from model-specific quirks. Together, these principles transform circuit discovery from exploratory data analysis into rigorous, replicable science.

We validate TinyLab by discovering the L0 hedging coalition in GPT-2—circuits providing mechanistic evidence consistent with Kalai et al.'s prediction that binary evaluation incentivizes trading factuality for hedging. The coalition survives every methodological test: it ranks in the 99\textsuperscript{th} percentile of 1,000~random layer-0 ablations, affects both accuracy ($\Delta$LD~$+30\%$) and calibration (ECE~$-25\%$), replicates across GPT-2 Small, Medium, and Mistral-7B despite architectural differences, and exhibits quantifiable mediation (67\% through layer-11 residual stream).

This work makes three contributions. \textbf{First}, we provide complete infrastructure for reproducible circuit discovery ($\sim$1,600~lines of code, full documentation, cross-platform support, seed packs, determinism verification). \textbf{Second}, we introduce methodological innovations (H1/H5/H6/H7~batteries, dual observables, random baseline enforcement, cross-architecture pipelines) that catch the narrow-sweep problem invalidating our prior work. \textbf{Third}, we demonstrate validation via the hedging coalition, proving TinyLab surfaces genuine behavioral circuits learned by gradient descent, not artifacts of ad-hoc analysis.

By treating mechanistic interpretability as an engineering discipline requiring standardized tooling—not just conceptual insight—TinyLab enables the systematic, reproducible science needed to understand and align increasingly powerful language models. The framework, datasets, and complete reproducibility package are available at \url{https://github.com/username/tinyLab}, designed for drop-in replication across model families and community extension.

Future work will expand model coverage (Llama, Pythia, Qwen), instrument training dynamics (when do coalitions emerge?), integrate SAE-based feature decomposition (H7~full implementation), and build an open database of circuits with standardized measurements. TinyLab provides the foundation for this circuit taxonomy, transforming one-off discoveries into cumulative, falsifiable knowledge about how transformers compute.

\appendix

\section{Lexicon and enrichment statistics}
\label{app:lexicon}

The hedge/booster lexicon used in Section~\ref{sec:findings} is stored at \texttt{data/lexicons/hedge\_booster.json}.
Tokens from the OV projections are normalised by stripping whitespace, punctuation, and byte-pair fragments before lookup.
We estimate enrichment by comparing the top-$150$ OV tokens for each suppressor head against the pool of other layer-0 heads with 1{,}000 frequency-matched resamples and add-$0.5$ smoothing.
Table~\ref{tab:lexicon-logodds} summarises the resulting log-odds ratios and the AUC of a single-feature classifier that predicts ``upweighted'' if a token is in the lexicon.

\begin{table}[h]
    \centering
    \caption{Lexicon enrichment for suppressor heads (top-$150$ OV tokens).}
    \label{tab:lexicon-logodds}
    \begin{tabular}{lccc}
        \toprule
        Head & Lexicon & Log-odds & AUC \\
        \midrule
        GPT-2\;0:2 & Hedges & $+1.22$ & $0.50$ \\
        GPT-2\;0:2 & Boosters & $+4.29$ & $0.52$ \\
        GPT-2\;0:4 & Hedges & $-1.27$ & $0.50$ \\
        GPT-2\;0:7 & Hedges & $+0.19$ & $0.50$ \\
        Mistral\;0:22/0:23 & Hedges/Boosters & $\approx 0$ & $0.50$ \\
        \bottomrule
    \end{tabular}
\end{table}

The enrichment confirms that GPT-2 head~0:2 amplifies both hedges and boosters relative to other layer-0 heads, whereas the remaining GPT-2 heads and the Mistral pair exhibit no measurable enrichment. The AUC values stay near 0.50, as expected for a single-feature sanity check.

\begin{table}[t]
    \centering
    \caption{Representative OV tokens for GPT-2 Medium head 0:2 (top/bottom five).}
    \label{tab:tokens-gpt-2-02}
    \begin{tabular}{@{}p{0.45\textwidth}p{0.45\textwidth}@{}}
        \toprule
        Raw BPE & Normalised word \\
        \midrule
        \multicolumn{2}{@{}l}{\textbf{Upweighted}} \\
        \midrule
        yne & yne \\
         totally\textsuperscript{B} & totally \\
         solid & solid \\
         advanced & advanced \\
         Kass & kass \\
        \midrule
        \multicolumn{2}{@{}l}{\textbf{Downweighted}} \\
        \midrule
        Recomm & recomm \\
         trave & trave \\
        accompan & accompan \\
         sacrific & sacrific \\
         advoc & advoc \\
        \bottomrule
    \end{tabular}
    \footnotesize{Top-$K$ tokens selected after frequency-matched resampling; see Section~\ref{sec:methods}.}
\end{table}

\begin{table}[t]
    \centering
    \caption{Representative OV tokens for GPT-2 Medium head 0:4 (top/bottom five).}
    \label{tab:tokens-gpt-2-04}
    \begin{tabular}{@{}p{0.45\textwidth}p{0.45\textwidth}@{}}
        \toprule
        Raw BPE & Normalised word \\
        \midrule
        \multicolumn{2}{@{}l}{\textbf{Upweighted}} \\
        \midrule
         Pik & pik \\
         Benz & benz \\
         Bud & bud \\
         Dem & dem \\
         Hobby & hobby \\
        \midrule
        \multicolumn{2}{@{}l}{\textbf{Downweighted}} \\
        \midrule
         streng & streng \\
         cryst & cryst \\
         notor & notor \\
         destro & destro \\
         nodd & nodd \\
        \bottomrule
    \end{tabular}
    \footnotesize{Top-$K$ tokens selected after frequency-matched resampling; see Section~\ref{sec:methods}.}
\end{table}

\begin{table}[t]
    \centering
    \caption{Representative OV tokens for GPT-2 Medium head 0:7 (top/bottom five).}
    \label{tab:tokens-gpt-2-07}
    \begin{tabular}{@{}p{0.45\textwidth}p{0.45\textwidth}@{}}
        \toprule
        Raw BPE & Normalised word \\
        \midrule
        \multicolumn{2}{@{}l}{\textbf{Upweighted}} \\
        \midrule
        ruciating & ruciating \\
         guiActiveUnfocused & guiactiveunfocused \\
         sights & sights \\
        atherine & atherine \\
         pag & pag \\
        \midrule
        \multicolumn{2}{@{}l}{\textbf{Downweighted}} \\
        \midrule
        theless & theless \\
        Redditor & redditor \\
         horizont & horizont \\
         condem & condem \\
        Ire & ire \\
        \bottomrule
    \end{tabular}
    \footnotesize{Top-$K$ tokens selected after frequency-matched resampling; see Section~\ref{sec:methods}.}
\end{table}

\begin{table}[t]
    \centering
    \caption{Representative OV tokens for Mistral 7B head 0:22 (top/bottom five).}
    \label{tab:tokens-mistral-022}
    \begin{tabular}{@{}p{0.45\textwidth}p{0.45\textwidth}@{}}
        \toprule
        Raw BPE & Normalised word \\
        \midrule
        \multicolumn{2}{@{}l}{\textbf{Upweighted}} \\
        \midrule
        giornata & giornata \\
        listade & listade \\
        revs & revs \\
        acknow & acknow \\
        occas & occas \\
        \midrule
        \multicolumn{2}{@{}l}{\textbf{Downweighted}} \\
        \midrule
        oppon & oppon \\
        LIED & lied \\
        itself & itself \\
        MVT & mvt \\
        recurs & recurs \\
        \bottomrule
    \end{tabular}
    \footnotesize{Top-$K$ tokens selected after frequency-matched resampling; see Section~\ref{sec:methods}.}
\end{table}

\begin{table}[t]
    \centering
    \caption{Representative OV tokens for Mistral 7B head 0:23 (top/bottom five).}
    \label{tab:tokens-mistral-023}
    \begin{tabular}{@{}p{0.45\textwidth}p{0.45\textwidth}@{}}
        \toprule
        Raw BPE & Normalised word \\
        \midrule
        \multicolumn{2}{@{}l}{\textbf{Upweighted}} \\
        \midrule
        acknow & acknow \\
        rieben & rieben \\
        départ & depart \\
        kat & kat \\
        rass & rass \\
        \midrule
        \multicolumn{2}{@{}l}{\textbf{Downweighted}} \\
        \midrule
        ionato & ionato \\
        altogether & altogether \\
        Pf & pf \\
        strict & strict \\
        atan & atan \\
        \bottomrule
    \end{tabular}
    \footnotesize{Top-$K$ tokens selected after frequency-matched resampling; see Section~\ref{sec:methods}.}
\end{table}


\section{Calibration and numerical stability}
\label{app:calibration}

Reliability diagrams in Figure~\ref{fig:calibration} use 10 bins and probabilities derived from the log-odds between target and foil tokens.
To avoid numerical overflow we clip logits to the range $[-20,20]$ before applying the softmax, a setting that does not materially change the reported metrics.

\section{Reproducibility checklist}
\label{app:repro}

\begin{itemize}
    \item \textbf{Models.} GPT-2 Medium (355\,M) via TransformerLens 2.16.1; Mistral-7B v0.1 via the same interface.
    \item \textbf{Hardware.} Apple M-series (M3 Max) with macOS; computations run in deterministic mode (no dropout, fixed seeds).
    \item \textbf{Datasets.} Single-token probe suite (stored under \texttt{lab/data/corpora}); frequency summaries in \texttt{reports/token\_frequency\_summary.json}.
    \item \textbf{Runs.} Config and data hashes for Table~\ref{tab:impact} appear in \texttt{paper/supplement/supplement.md}; seeds are $\{0,1,2\}$ for GPT-2 and $\{0,1,2\}$ (neg/cf) / $\{0\}$ (facts/logic) for Mistral.
    \item \textbf{Commands.} \texttt{python -m lab.src.orchestrators.conditions <config>} (see Table~\ref{tab:impact} for the specific JSON files).
    \item \textbf{Figures.} Scripts in \texttt{paper/scripts/} regenerate the figures.
\end{itemize}

\section{Discovery path and transparency}\label{app:transparency}
We formulated a layer~0 prediction prior to targeted experiments based on bottleneck theory and Kalai et~al.'s constraint. We then ran fixed protocols: H1 head sweeps, random layer~0 baselines (1{,}000 resamples), H5 cooperation tests, H6 path mediation, and cross-architecture replication. We did not perform an exhaustive post-hoc search. A timestamped research log is available upon request.


\bibliographystyle{plain}
\bibliography{references}

\end{document}
